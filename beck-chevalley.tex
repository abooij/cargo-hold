\documentclass{article}

\usepackage[utf8]{inputenc}
\usepackage[safeinputenc,giveninits=true]{biblatex}
\usepackage{hyperref}
\usepackage{todonotes}
\usepackage{amsmath}
\usepackage{amsthm}
\usepackage{amssymb}
\usepackage{amsfonts}
\usepackage{xspace}
\usepackage{mathtools}
\usepackage[all]{xy}
\usepackage{pdfpages}
\usepackage{cleveref}
\usepackage{tikz-cd}

\newtheorem{proposition}{Proposition}

\title{Beck-Chevalley condition: proof details}
\date{October 2016}
\author{Auke Booij}
\begin{document}

\maketitle

\begin{proposition}
  Let the following square be a pullback square in a weak topos with
  $g$ and $h$ monic.

  \begin{tikzcd}
    A
      \arrow[r,"f"]
      \arrow[d,"g",rightarrowtail]
      \arrow[dr, phantom, "\lrcorner", very near start]
    &
    B
      \arrow[d,"h",rightarrowtail]
    \\
    C
      \arrow[r,"k"]
    &
    D
  \end{tikzcd}

  Then the following diagram (the Beck-Chevalley square) commutes.

  \begin{tikzcd}
    PA
      \arrow[d,"\exists g"]
      \arrow[dr, phantom, "\bigcirc" pos=0.6]
    &
    PB
      \arrow[l,"Pf"]
      \arrow[d,"\exists h"]
    \\
    PC
    &
    PD
      \arrow[l,"Pk"]
  \end{tikzcd}
\end{proposition}

\begin{proof}
  We show that the Beck-Chevalley square both ways round names the
  vertical composition $E_f \rightarrowtail PB \times A
  \overset{1\times g}{\rightarrowtail} PB \times C$ in the following
  diagram.

  \begin{tikzcd}[row sep=large]
    E_f
      \arrow[r]
      \arrow[d, "\tilde{f}", rightarrowtail]
    &
    \in_B
      \arrow[d, rightarrowtail]
    \\
    PB \times A
      \arrow[r,"1 \times f"]
      \arrow[d,"1 \times g", rightarrowtail]
    &
    PB \times B
      \arrow[d,"1 \times h", rightarrowtail]
    \\
    PB \times C
      \arrow[r,"1 \times k"]
    &
    PB \times D
  \end{tikzcd}

  % The lower square is a pullback (by assumption), and so is the upper
  % (by definition of $E_f$), hence the outer rectangle is.

  \medskip

  We will first show that the composite $(1 \times g) \circ \tilde{f}$ is named by
  $\exists g \circ Pf$.
  To show this, we have to show that the red outer square in the
  following diagram is a pullback.
  Here, the small squares $I$ and $III$ are pullbacks by definition.
  If $II$ were also a pullback, we could use pullback pasting, but
  alas this merely commutes.

  \begin{tikzcd}[row sep=large]
    E_f
      \arrow[r, red]
      \arrow[d, "\tilde{f}", red, rightarrowtail]
      \arrow[dr, phantom, "\lrcorner", very near start]
      \arrow[dr, phantom, "I"]
    &
    \in_A
      \arrow[r, red]
      \arrow[d, rightarrowtail]
      \arrow[ddr, phantom, "\lrcorner", very near start]
      \arrow[ddr, phantom, "III"]
    &
    \in_C
      \arrow[dd, red, rightarrowtail]
    \\
    PB \times A
      \arrow[r, "Pf \times 1"]
      \arrow[d, "1 \times g", red, rightarrowtail]
      \arrow[dr, phantom, "II"]
    &
    PA \times A
      \arrow[d, "1 \times g", rightarrowtail]
    &
    \\
    PB \times C
      \arrow[r, "Pf \times 1", red]
    &
    PA \times C
      \arrow[r, "\exists g \times 1", red]
    &
    PC \times C
  \end{tikzcd}

  We prove its universal property directly.
  So let $X$ be a cone.

  \begin{tikzcd}[row sep=large]
    X
      \arrow[rddd, "l", bend right]
      \arrow[rrrd, "m", bend left=15]
      \arrow[rrd, "n", dotted]
      \arrow[rdd, "o" description, dotted]
      \arrow[rd, dotted]
    & & & \\
    &
    E_f
      \arrow[r, red]
      \arrow[d, "\tilde{f}", red, rightarrowtail]
      \arrow[dr, phantom, "\lrcorner", very near start]
      \arrow[dr, phantom, "I"]
    &
    \in_A
      \arrow[r, red]
      \arrow[d, rightarrowtail]
      \arrow[ddr, phantom, "\lrcorner", very near start]
      \arrow[ddr, phantom, "III"]
    &
    \in_C
      \arrow[dd, red, rightarrowtail]
    \\
    &
    PB \times A
      \arrow[r, "Pf \times 1"]
      \arrow[d, "1 \times g", red, rightarrowtail]
      \arrow[dr, phantom, "II"]
    &
    PA \times A
      \arrow[d, "1 \times g", rightarrowtail]
    &
    \\
    &
    PB \times C
      \arrow[r, "Pf \times 1", red]
    &
    PA \times C
      \arrow[r, "\exists g \times 1", red]
    &
    PC \times C
  \end{tikzcd}

  By the fact that square $III$ is a pullback, we obtain
  the map $n$ via the maps $m$ and $(Pf \times 1) \circ l$.

  The map $o$ is defined as:
  \[
    o := \left ( \pi_0 \circ l , \pi_1 \circ \in_A \circ n \right )
  \]
  The fact that $(o,n)$ form a cone for pullback square $I$ is then
  easily verified.
  Hence we obtain the required map $X \to E_f$, which is a morphism of
  cones for the outer square (check!).

  To show that it is unique, suppose that there is another morphism of
  cones.
  Then their compositions along the left vertical (i.e.\ along $(1
  \times g) \circ \tilde{f}$) are both equal to $l$.
  But $(1 \times g) \circ \tilde{f}$ is monic, so the maps are equal.

  \medskip

  We will now show that $(1 \times g) \circ \tilde{f}$ is named by
  $Pk \circ \exists h$.
  This amounts to showing that the following outer red square is a
  pullback.

  \begin{tikzcd}[column sep=small, row sep=small]
    &
    \in_B
      \arrow[rr]
      \arrow[dd, rightarrowtail]
      \arrow[ddddrr, phantom, "III", very near start]
    &&
    \in_D
      \arrow[dddd, rightarrowtail]
      \arrow[dddddl, phantom, "IV", pos=0.08]
    &
    \\
%
%
    E_f
      \arrow[ur]
      \arrow[rr, dotted, red, crossing over]
      \arrow[dd, "\tilde{f}", red, rightarrowtail]
      \arrow[dr, phantom, "I"]
    &&
    E_k
      \arrow[ur]
      \arrow[rr, red, crossing over]
    &&
    \in_C
      \arrow[dddd, red, rightarrowtail]
      \arrow[ddddll, phantom, "V", very near start]
    \\
%
%
    &
    PB \times B
      \arrow[dd, "1 \times h", rightarrowtail]
    &&&
    \\
%
%
    PB \times A
      \arrow[ur, "1 \times f"]
      \arrow[dd, "1 \times g", red, rightarrowtail]
      \arrow[dr, phantom, "II"]
    &&&&
    \\
%
%
    &
    PB \times D
      \arrow[rr, "\exists h \times 1" near start]
    &&
    PD \times D
    &
    \\
%
%
    PB \times C
      \arrow[ur, "1 \times k"]
      \arrow[rr, "\exists h \times 1", red]
    &&
    PD \times C
      \arrow[ur, "1 \times k"]
      \arrow[rr, "Pk \times 1", red]
      \arrow[uuuu, "\tilde{k}"', crossing over, leftarrowtail]
    &&
    PC \times C
  \end{tikzcd}

  The small (non-composed) squares $I$, $III$, $IV$ and $V$ are
  pullbacks by definition.
  Square $II$ is a pullback by assumption.

  The map $E_f \to E_k$ exists because square $IV$ is a
  pullback. (exercise: how do we make $E_f$ into a cone?)

  To show that the red square is a pullback, let $X$ be an arbitrary
  cone over it.

  \begin{tikzcd}[column sep=small, row sep=small]
    X
      \arrow[ddddddr, "l", bend right]
      \arrow[rrrrrdd, "m", bend left]
      \arrow[rrrdd, "n" near start, dotted, end anchor=north west]
      \arrow[rrd, "o", dotted, bend left]
      \arrow[rdd, dotted]
    &&&&&
    \\
%
%
    &&
    \in_B
      \arrow[rr]
      \arrow[dd, rightarrowtail]
      \arrow[ddddrr, phantom, "III", very near start]
    &&
    \in_D
      \arrow[dddd, rightarrowtail]
      \arrow[dddddl, phantom, "IV", pos=0.08]
    &
    \\
%
%
    &
    E_f
      \arrow[ur]
      \arrow[rr, dotted, red, crossing over]
      \arrow[dd, "\tilde{f}", red, rightarrowtail]
      \arrow[dr, phantom, "I"]
    &&
    E_k
      \arrow[ur]
      \arrow[rr, red, crossing over]
    &&
    \in_C
      \arrow[dddd, red, rightarrowtail]
      \arrow[ddddll, phantom, "V", very near start]
    \\
%
%
    &&
    PB \times B
      \arrow[dd, "1 \times h", rightarrowtail]
    &&&
    \\
%
%
    &
    PB \times A
      \arrow[ur, "1 \times f"]
      \arrow[dd, "1 \times g", red, rightarrowtail]
      \arrow[dr, phantom, "II"]
    &&&&
    \\
%
%
    &&
    PB \times D
      \arrow[rr, "\exists h \times 1" near start]
    &&
    PD \times D
    &
    \\
%
%
    &
    PB \times C
      \arrow[ur, "1 \times k"]
      \arrow[rr, "\exists h \times 1", red]
    &&
    PD \times C
      \arrow[ur, "1 \times k"]
      \arrow[rr, "Pk \times 1", red]
      \arrow[uuuu, "\tilde{k}"', crossing over, leftarrowtail]
    &&
    PC \times C
  \end{tikzcd}

  $n$ is obtained by pullback square $V$ from $m$ and $(\exists h
  \times 1) \circ l$.
  So we obtain $o$ by pullback square $III$.
  Hence, because $I$ and $II$ are pullbacks, we obtain the required
  map $X \to E_f$.

  Just like before, uniqueness of that map follows from the fact that
  $(1 \times g) \circ \tilde{f}$ is monic.

  \medskip

  Hence, because names of subobjects are unique, we get the required commutativity.
\end{proof}



\end{document}

%%% Local Variables:
%%% mode: latex
%%% TeX-master: t
%%% End:
